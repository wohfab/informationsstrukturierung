\chapter{Schluss-Erklärung und Ausblick}
\label{chap:Schluss-Erklärung und Ausblick}

% Ergebnisse zusammenfassen und bewerten - Beantworten der Fragen aus der Einleitung - Ausblick / Offene Fragen / Angrenzende Themengebiete
% KEINE neuen Erkenntnisse / Thesen

Wie in \ref{section:weiteredateien} beschrieben, wird aus der Datenbank \texttt{datenbank.xml} und dem Stylesheet \texttt{stylesheet.xsl} mithilfe der \texttt{XSL} Transformation die Datei \texttt{index.html} generiert.

Die Transformation lässt sich beispielsweise mit dem Programm \texttt{xsltproc} (\href{http://xmlsoft.org/XSLT/xsltproc.html}{Dokumentation unter xmlsoft.org/XSLT/xsltproc.html}) über die Kommandozeile vornehmen.

Befindet man sich mit der Kommandozeile im Wurzelverzeichnis dieses Projektes, ist der Transformations-Befehl wiefolgt:

\begin{lstlisting}[language=bash, caption=\texttt{XSL} Transformation]
xsltproc -o index.html stylesheet.xsl datenbank.xml
\end{lstlisting}

Zum Abschluss dieses Projektes möchte ich einen Ausblick geben über mögliche Erweiterungen dieses Projektes.

Die Erweiterung der Datenbank um weitere Cocktails liegt auf der Hand. Zur Veranschaulichung der eingebauten Funktionalitäten befinden sich momentan zwölf Cocktail-Einträge und die dafür nötigen Zutaten in der Datenbank. Über eine Automatisierung ließen sich entsprechend auch größere Mengen an Cocktail-Rezepten in der Datenbank einpflegen.

Eine Rechnung, die man in einem weiteren Projekt einbauen könnte, wäre ein Alkohol-Messer. Wenn man die Zutaten-Einträge in der Datenbank um den Wert des Alkohol-Gehaltes erweitert, so ließe sich aus der Gesamtmenge der Flüssigkeit innerhalb eines Cocktails und den entsprechenden Alkohol-Werten der Gesamt-Alkohol eines Cocktails errechnen.