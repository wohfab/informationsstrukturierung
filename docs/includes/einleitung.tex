\chapter{Einleitung}
\label{chap:einleitung}

% Zielsetzung - Problemstellung - Eingrenzung/Abgrenzung des Themas (begründet) - Aufbau - Roter Faden

\section{Grundlage}

Grundlage des vorliegenden Projektes sind das Modul \texttt{23-TXT-BaCL4} im Nebenfach Texttechnologie und Computerlinguistik und das dazugehörige Skript von Prof. Dr. Marcus Kracht \cite{kracht2018}.

Im Rahmen der zwei Seminare \textit{Informationsstrukturierung} \& \textit{Informationsstrukturierung 2} wurde gelernt, wie eine Datenbank mithilfe von \texttt{XML} angelegt und unter Zunahme eines \texttt{XML Schemas} strukturiert werden kann. Über eine \texttt{XSL}, eine sogenannte Stylesheet-Datei, kann aus der Datenbank eine \texttt{HTML} Datei erzeugt werden. Dieser Prozess ist bekannt unter dem Namen \texttt{XSLT}; der sogenannten \textit{Transformation}. Diese kann über die \texttt{HTML} Datei noch eine Vielzahl anderer Dateiformate erzeugen, denen wir uns jedoch in dieser Ausarbeitung nicht widmen.

\section{Projekteinführung}

Bei dem vorliegenden Projekt handelt es sich um die \textit{digitale Cocktailbar}; ein Rezepteverzeichnis für Cocktails. Eine Sammlung an Cocktailrezepten wird in einer Datenbank gespeichert, aus der mithilfe der oben genannten Transformations-Anwendung eine Website erzeugt wird, die im Web-Browser angeschaut werden kann.

Nach dieser kurzen Einleitung folgt in Kapitel \ref{chap:dateien} eine detaillierte Beschreibung der im Projekt vorhandenen Dateien und deren Inhalten und Aufgaben.


